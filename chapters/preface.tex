\chapter{Preface}

\textbf{Note:} this book is currently a work in progress. Take this preface as
a sort of manifesto on its contents.

This book aims to provide a complete introduction and guide to applied
statistics for a Physics student, from freshman year to the Masters' thesis,
alternating between first principles and concrete solutions to messy real world
situations. The focus is on students mainly interested in experimental Physics,
but the techniques also apply to the analysis of theoretical numerical
calculations.

The subject is presented exclusively from the point of view of Bayesian
statistics. The more common frequentist approach is introduced only towards the
end, in \autoref{ch:5}, but only to provide a bridge---a way of understanding
information written in the frequentist language and translating it to the
Bayesian paradigm. Although there is no formal context in which to ``prove''
which of the two alternatives (or any other) is the correct one, experience and
reflexion over the years within both worldviews have led me to conclude that
the Bayesian way is the most efficient one, in the sense of reducing the time
required to learn statistics in the first place, the amount of concepts needed
to start dealing with real-world problems, the effort spent to start a first
attack at an analysis, etc., due to to an higher degree of simplicity and
coherence. If you are acquainted with the frequentist toolset, don't worry:
despite these conceptual benefits, most of the time the calculations will be
the same, just interpreted under a different light. Nothing really unusual will
happen. There is always a way to ``backport'' any specific solution between the
two methods---the difference is that the path through Bayes is shorter and
straighter.

We do not provide ``serious'' exercises at the end of each chapter, only simple
ones intermixed with the theory. This is because we think it would be difficult
to cast as a contained exercise all the richness and inexpected intricacy of
real data. This book assumes that the reader, at any moment, is following a
laboratory course, as is standard in Physics curricula. The content of this
book either proves its worth in the lab, or is fated to serve no purpose.

The lighter exercises are meant to help the reader fixate new concepts by
actively deriving basic properties instead of passively reading proofs. By all
means, if you can't solve an exercise, or have even a shred of doubt about your
procedure, read the solution in \autoref{ch:sol}. The exercises are not
optional.

The material presented here mostly stems from my everyday (unsolicited) studies
while going through my Physics degree. However, I would have probably never
considered Bayesian statistics if it wasn't for a weird and intelligent fellow
student, Valerio Lomanto, who in turns owes this insight to another weird and
intelligent man, a Jew of the internet going by the name of Yudkowsky. For
anything which is not my own concoction, the following books are my references:
%
\begin{itemize}
    %
    \item Jaynes, \emph{Probability Theory: the Logic of Science},
    %
    \item Pearl, \emph{Causality: Models, Reasoning and Inference},
    %
    \item MacKay, \emph{Information Theory, Inference, and Learning Algorithms},
    %
    \item Gelman, \emph{Bayesian Data Analysis}.
    %
\end{itemize}
%
If you ever read Jaynes and Pearl, you will find that the explanations are
interspersed with conflictual statements against ``them'' or ``the
establishment''---sort of. Don't worry, they are no quacks; they are both well
respected scientists. This is a trope of Bayesian literature, since they can
not wrap their head around how it happened that, at the dawn of the XX century,
Bayes was dethroned as the rightful way to inference. I will follow the
tradition.
%
\begin{flushright}
    %
    Giacomo Petrillo, January 10, 2022
    %
\end{flushright}

\section*{License}

Along with the digital copy, the source code of this book---both for
typesetting the text and for generating the figures and examples---is released
online at
%
\begin{center}
    %
    \url{https://github.com/Gattocrucco/bayestat}
    %
\end{center}
%
under the open source Creative Commons license
\href{https://creativecommons.org/licenses/by-sa/4.0/}{Attribution-ShareAlike~4.0},
which grants anyone the right to freely copy, modify and share the book,
provided they cite the original source, highlight the eventual presence of
modifications, and do not imply endorsement by the author if not given
explicitly.
